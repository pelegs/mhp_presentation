\documentclass{beamer}
\usetheme{Goettingen}
\usefonttheme[onlymath]{serif}

\usepackage[version=4]{mhchem}
\usepackage{graphicx}
\usepackage{svg}
\usepackage{empheq}
\usepackage[many]{tcolorbox}

\usefonttheme{professionalfonts}
\usepackage{times}
\usepackage{amsmath}
\usepackage{verbatim}

\usepackage{tikz}
\usetikzlibrary{arrows,shapes,positioning,shadows,trees,matrix,math,ipe}
\tikzset{>=stealth}
\newcommand{\tikzmark}[3][]{\tikz[remember picture,baseline] \node [anchor=base,#1,](#2) {#3};}
\tikzstyle{ipe stylesheet} = [
  ipe import,
  even odd rule,
  line join=round,
  line cap=butt,
  ipe pen normal/.style={line width=0.4},
  ipe pen heavier/.style={line width=0.8},
  ipe pen fat/.style={line width=1.2},
  ipe pen ultrafat/.style={line width=2},
  ipe pen normal,
  ipe mark normal/.style={ipe mark scale=3},
  ipe mark large/.style={ipe mark scale=5},
  ipe mark small/.style={ipe mark scale=2},
  ipe mark tiny/.style={ipe mark scale=1.1},
  ipe mark normal,
  /pgf/arrow keys/.cd,
  ipe arrow normal/.style={scale=7},
  ipe arrow large/.style={scale=10},
  ipe arrow small/.style={scale=5},
  ipe arrow tiny/.style={scale=3},
  ipe arrow normal,
  /tikz/.cd,
  ipe arrows, % update arrows
  <->/.tip = ipe normal,
  ipe dash normal/.style={dash pattern=},
  ipe dash dashed/.style={dash pattern=on 4bp off 4bp},
  ipe dash dotted/.style={dash pattern=on 1bp off 3bp},
  ipe dash dash dotted/.style={dash pattern=on 4bp off 2bp on 1bp off 2bp},
  ipe dash dash dot dotted/.style={dash pattern=on 4bp off 2bp on 1bp off 2bp on 1bp off 2bp},
  ipe dash normal,
  ipe node/.append style={font=\normalsize},
  ipe stretch normal/.style={ipe node stretch=1},
  ipe stretch normal,
  ipe opacity 10/.style={opacity=0.1},
  ipe opacity 30/.style={opacity=0.3},
  ipe opacity 50/.style={opacity=0.5},
  ipe opacity 75/.style={opacity=0.75},
  ipe opacity opaque/.style={opacity=1},
  ipe opacity opaque,
]

\definecolor{red}{rgb}{1,0,0}
\definecolor{green}{rgb}{0,1,0}
\definecolor{blue}{rgb}{0,0,1}
\definecolor{yellow}{rgb}{1,1,0}
\definecolor{orange}{rgb}{1,0.647,0}
\definecolor{gold}{rgb}{1,0.843,0}
\definecolor{purple}{rgb}{0.627,0.125,0.941}
\definecolor{gray}{rgb}{0.745,0.745,0.745}
\definecolor{brown}{rgb}{0.647,0.165,0.165}
\definecolor{navy}{rgb}{0,0,0.502}
\definecolor{pink}{rgb}{1,0.753,0.796}
\definecolor{seagreen}{rgb}{0.18,0.545,0.341}
\definecolor{turquoise}{rgb}{0.251,0.878,0.816}
\definecolor{violet}{rgb}{0.933,0.51,0.933}
\definecolor{darkblue}{rgb}{0,0,0.545}
\definecolor{darkcyan}{rgb}{0,0.545,0.545}
\definecolor{darkgray}{rgb}{0.663,0.663,0.663}
\definecolor{darkgreen}{rgb}{0,0.392,0}
\definecolor{darkmagenta}{rgb}{0.545,0,0.545}
\definecolor{darkorange}{rgb}{1,0.549,0}
\definecolor{darkred}{rgb}{0.545,0,0}
\definecolor{lightblue}{rgb}{0.678,0.847,0.902}
\definecolor{lightcyan}{rgb}{0.878,1,1}
\definecolor{lightgray}{rgb}{0.827,0.827,0.827}
\definecolor{lightgreen}{rgb}{0.565,0.933,0.565}
\definecolor{lightyellow}{rgb}{1,1,0.878}
\definecolor{black}{rgb}{0,0,0}
\definecolor{white}{rgb}{1,1,1}
\tcbset{highlight math style={enhanced, colframe=blue!20!black, colback=white, arc=4pt, boxrule=0.5pt}}

\title[Molecular Hydrophobicity Potential]
{%
    Implementing Molecular Hydrophobicity Potential Measurment for the Analysis of Dynamic Biomolecular Interactions
}
\date{\today}
\author[Pelg Bar Sapir]
{
    Peleg Bar Sapir\inst{1} \and \\
    Under supervision of Prof. Maria Andrea Mroginski\inst{2}
}
\institute[Freie Universit\"{a}t Berlin, Techniche Universit\"{a} Berlin]
{
    \inst{1}%
        Freie Universit\"{a}t Berlin
    \and
    \vskip-2mm
    \inst{2}%
        Techniche Universit\"{a}t Berlin
}

\begin{document}
\tikzstyle{every picture}+=[remember picture]
\everymath{\displaystyle}

\begin{frame}
    \titlepage
\end{frame}

\begin{frame}{Outline}
    \tableofcontents
\end{frame}

\section{Introduction}
\subsection{Hydrophobicity and log P}

\section{Molecular Hydrophobicity Potential}
\subsection{Potential}
\subsubsection{General form}

\begin{frame}{The MHP Formula}

    \centering
    \includesvg[scale=0.1]{mhp_visual1}

    \begin{equation*}
        \text{MHP}\left(\mathbf{x'}\right)=
        \tikz[baseline]{
            \node[anchor=base] (sum)
            {$\sum_{i=1}^{k}\limits$};
        }\left[ 
        \tikz[baseline]{
            \node[anchor=base] (force)
            {$f_i$};
        } \cdot
        \tikz[baseline]{
            \node[anchor=base] (distance)
            {$D\left(\mathbf{x}-\mathbf{x'}_i\right)$};
        }\right]
    \end{equation*}

    \begin{tikzpicture}[overlay, remember picture,node distance=1.5cm]
        \uncover<2->{\node[fill=blue!20, xshift=1.2cm, yshift=0.5cm] (sumdescr) [below left=of sum]{Summing over all atoms};}
        \uncover<2->{\draw[->,thick] (sumdescr) to [in=-90,out=90] (sum);}
        \uncover<2->{\node[fill=blue!20] at (sum) {$\sum_{i=1}^{k}\limits$};}
                        
        \uncover<3->{\node[fill=red!20, yshift=-0.2cm] (forcedescr) [below = of force]{Force constants};}
        \uncover<3->{\draw[->,thick] (forcedescr) to [in=-90,out=90] (force);}
        \uncover<3->{\node[fill=red!20] at (force) {$f_i$};}

        \uncover<4->{\node[fill=green!20, xshift=-2.0cm,yshift=0.5cm] (distancedescr) [below right = of distance]{Distance function};}
		\uncover<4->{\draw[->,thick] (distancedescr) to [in=-90,out=90] (distance.south);}
        \uncover<4->{\node[fill=green!20] at (distance) {$D\left(\mathbf{x}-\mathbf{x'}_i\right)$};}
	\end{tikzpicture}
\end{frame}

\subsubsection{Force constants}
\begin{frame}{Force constants}
    \centering
    \begin{tabular}{l l r}
        Type & Description & $f_i$ value \\
        \hline
            & \underline{C in:} &         \\
        3   & $\ce{CHR_3}$      & -0.6681 \\
        15  & $\ce{=CH_2}$      & -0.7866 \\
        36  & $\ce{R-CH-X}$     & -0.2405 \\
            & & \\
            & \underline{H attached to}:                      &         \\
        45  & $\ce{C_{sp^{3}}}$, no X attached to next carbon &  0.7341 \\
        46  & $\ce{C_{sp^{3}}, C_{sp^{2}}}$                   &  0.6301 \\
        50  & Heteroatom                                      & -0.1036 \\
        52  & $\ce{C_{sp^{3}}}$, 1 X attached to next carbon  &  0.6666 \\
            & & \\
            & \underline{O in}: &         \\
        56  & Alcohol           & -0.3567 \\
        58  & Ketone            & -0.0233 \\
        62  & \ce{O-}           & -0.7941 \\
        \hline
    \end{tabular}
    ~\\
    \tiny{Source: Arup K. Ghose et al, J. Phys. Chem. A 1998, 102, 3762-3772}
\end{frame}

\subsubsection{Distance function}
\begin{frame}{Distance function}
    \centering
    \begin{minipage}[t]{0.48\linewidth}
        \centering
        Audry form
        \begin{empheq}[box=\tcbhighmath]{align*}
            D\left(x\right)=\frac{1}{1+x}
        \end{empheq}
    \end{minipage}
    \begin{minipage}[t]{0.48\linewidth}
        \centering
        Exponential decay form
        \begin{empheq}[box=\tcbhighmath]{align*}
            D\left(x\right)=e^{-\alpha x}
        \end{empheq}
    \end{minipage}
    \includegraphics[scale=0.65]{dist_funcs.pdf}    
\end{frame}

\subsection{Surface}
\subsubsection{Solvent accesible surface}
\begin{frame}{Solvent accesible surface}
	\begin{itemize}
		\item The surface around a molecule accesible to solvent molecules \\
		\uncover<2->{	
			\begin{tikzpicture}[ipe stylesheet, scale=0.75]
				\filldraw[ipe pen fat, ipe dash dashed, fill=lightblue]
				(224, 672) circle[radius=54.5026];
				\filldraw[ipe pen fat, ipe dash dashed, fill=lightblue]
				(255.5303, 627.0014)
				arc[start angle=-128.6284, end angle=123.5534, radius=51.8715];
				\filldraw[ipe pen fat, ipe dash dashed, fill=lightblue]
				(168, 680.0003)
				arc[start angle=90.7993, end angle=364.8178, radius=63.4375];
				\filldraw[ipe pen fat, fill=red]
				(224, 672) circle[radius=32];
				\filldraw[ipe pen fat, fill=white]
				(288, 668) circle[radius=24];
				\draw[-]
				(312, 668) -- (340,668);
				\node[ipe node, text=black]
				at (322, 672) {$r$};
				\filldraw[ipe pen fat, fill=gray]
				(168, 616) circle[radius=40];
				\filldraw[ipe pen heavier, fill=purple!20]
				(230.967, 616.9735) circle[radius=21.7552];
				\pic[ipe mark large, purple]
				at (230.968, 616.973) {ipe disk};
				\draw[purple, ipe pen heavier, ->]
				(230.9679, 616.973)
				-- (217.7959, 634.2885);
				\node[ipe node, text=purple]
				at (228.134, 625.542) {$\vec{r}$};
				
				% Legend
				\draw[ipe pen fat] (320, 520) rectangle ++(140,80);
				\filldraw[ipe pen fat, ipe dash dashed, fill=lightblue]
				(330,570) rectangle ++(20, 20);
				\node[ipe node, text=black]
				at (360, 575) {SAS};
				\filldraw[ipe pen heavier, fill=red]
				(340,550) arc (90:270:10);
				\filldraw[ipe pen heavier, fill=white]
				(340,550) arc (90:-90:10);
				\node[ipe node, text=black]
				at (360, 535) {VdW surface};
			\end{tikzpicture}
		}
	\uncover<3->{\item For water molecules usually $r=4\left[\AA\right]$}
	\end{itemize}
\end{frame}

\subsubsection{Evenly distributed points}
\begin{frame}{Evenly distributed points}
    \centering
    How to distribute $N$ points on a surface of a sphere?\\ ~\\ ~\\
    \begin{columns}
        \uncover<1->{
        \begin{column}{0.4\textwidth}
        \centering
	        \includesvg[scale=.17]{sphere_wireframe_coords}\\
        \end{column}
        }
        
        \uncover<3->{
        \begin{column}{0.2\textwidth}
        \centering
	        \includesvg[scale=.17]{arrow_right}
        \end{column}
        \begin{column}{0.4\textwidth}
        \centering
	        \includesvg[scale=.17]{sphere_wireframe_points}
        \end{column}
    }
    \end{columns}
    \begin{columns}
        \begin{column}{0.4\textwidth}
            \begin{itemize}
                \uncover<2->{\item[] $\varphi_{i}=i\cdot\frac{2\pi}{N}$}
                \uncover<2->{\item[] $\theta_{j}=j\cdot\frac{\pi}{N}$}
            \end{itemize}
        \end{column}
        \begin{column}{0.2\textwidth}
        \end{column}
        \begin{column}{0.4\textwidth}
            \begin{itemize}
                \uncover<4->{\item Points are not evenly distributed}
                \uncover<5->{\item Several points overlap at poles}
            \end{itemize}
        \end{column}
    \end{columns}
\end{frame}

\begin{frame}{Evenly distributed points}
    \centering
    Solution: \alert{Vogel's method}\\
    \begin{columns}
        \begin{column}{0.5\textwidth}
            \uncover<1->{In 2 dimensions:}
            \begin{itemize}
                \uncover<2->{\item Distances: $r_{i}=\sqrt{\frac{i}{N}}$}
                \uncover<2->{\item Angle: $\theta_{i}=\varphi i$
                			 \item[] ($\varphi$ is the golden ratio!)}
            \end{itemize}
        \end{column}
        \begin{column}{0.5\textwidth}
			\centering
			\uncover<3->{\includesvg[scale=0.35]{vogels}}
        \end{column}
    \end{columns}
    ~\\ ~\\
    \begin{columns}
        \begin{column}{0.5\textwidth}
            \uncover<4->{In 3 dimensions (cylindrical coordinates):}
            \begin{itemize}
                \uncover<5->{\item Distances: $z_{i}=\left(1-\frac{1}{N}\right)\left(1-\frac{2i}{N-1}\right)$}
                \uncover<5->{\item Angles: $\theta_{i}=\varphi i,\ \rho_{i}=\sqrt{1-z_{i}^{2}}$}
            \end{itemize}
        \end{column}
        \begin{column}{0.5\textwidth}
			\centering
			\uncover<6->{\includegraphics[scale=.4]{3d_vogels.png}\\
                         \tiny{Image source: Marmakoide's Blog}
            }
        \end{column}
    \end{columns}
\end{frame}

\subsubsection{Integration}
\begin{frame}{Integration}
\end{frame}
\end{document}
